\documentclass[a4paper, 11pt, notoc]{article} % notoc
\pdfoutput=1
\usepackage{jcappub}
\notoc
\usepackage{graphicx}
\usepackage{booktabs}
\usepackage{verbatim}
\usepackage{caption}
\usepackage{xspace}
\usepackage{hyperref}
\usepackage{multirow}
\usepackage{placeins}
\usepackage{array}
\usepackage{subcaption}

% Variables
\newcommand{\sigv}{\ensuremath{\langle \sigma v_{\rm{rel}} \rangle}\xspace}
\newcommand{\MET}{\ensuremath{E_T^\mathrm{miss}}\xspace}
\newcommand{\met}{\MET}
\newcommand{\MT}{\ensuremath{M_{T}}\xspace}
\newcommand{\pt}{\ensuremath{p_{T}}\xspace}

% Units
\newcommand{\GeV}{\textrm{GeV}\xspace}
\newcommand{\gev}{\GeV\xspace}
\newcommand{\TeV}{\textrm{TeV}\xspace}
\newcommand{\tev}{\TeV\xspace}

% Particle names
\newcommand{\lp}{\ensuremath{l^{+}}\xspace}
\newcommand{\lm}{\ensuremath{l^{-}}\xspace}
\newcommand{\ttbar}{\ensuremath{\bar{t}t}}
\newcommand{\bbbar}{\ensuremath{\bar{b}b}}
\newcommand{\A}{A}
\newcommand{\pa}{a}
\newcommand{\pH}{H}
\newcommand{\pZ}{Z}
\newcommand{\Hc}{\ensuremath{H^{\pm}}\xspace}

% Particle masses
\newcommand{\mDM}{\ensuremath{M_{\chi}}\xspace}
\newcommand{\mdm}{\ensuremath{M_{\chi}}\xspace}
\newcommand{\mmed}{\ensuremath{M_{\rm{med}}}\xspace}
\newcommand{\mMed}{\ensuremath{M_{\rm{med}}}\xspace}
\newcommand{\mZ}{\ensuremath{M_{\rm{Z}}}\xspace}
\newcommand{\mA}{\ensuremath{M_{A}}\xspace}
\newcommand{\ma}{\ensuremath{M_{a}}\xspace}
\newcommand{\mH}{\ensuremath{M_{H}}\xspace}
\newcommand{\mHc}{\ensuremath{M_{H^{\pm}}}\xspace}
\newcommand{\mh}{\ensuremath{M_{h}}\xspace}
\newcommand{\mt}{\ensuremath{M_{t}}\xspace}

% Couplings
\newcommand{\gDM}{\ensuremath{g_{\rm{DM}}}\xspace}
\newcommand{\gq}{\ensuremath{g_q}\xspace}
\newcommand{\gSM}{\gq}
\newcommand{\gdm}{\gDM}
\newcommand{\ifb}{\ensuremath{\rm{fb}^{-1}}\xspace}

% Other parameters
\newcommand{\sinp}{\ensuremath{\sin\theta}\xspace}
\newcommand{\cosp}{\ensuremath{\cos\theta}\xspace}
\newcommand{\sinbma}{\ensuremath{\sin(\beta - \alpha)}\xspace}
\newcommand{\cosbma}{\ensuremath{\cos(\beta - \alpha)}\xspace}
\newcommand{\tanb}{\ensuremath{\tan\beta}\xspace}
\newcommand{\lap}[1]{\lambda_{P#1}} % can use like \lap1 , \lap2
\newcommand{\lam}[1]{\lambda_{#1}} % can use like \lam3

% Search channels
\newcommand{\metplusx}{\ensuremath{\MET+X}\xspace}
\newcommand{\hdm}{\ensuremath{h+\textrm{DM}}\xspace}
\newcommand{\monoh}{\ensuremath{h+\MET}\xspace}
\newcommand{\monohbb}{\ensuremath{h(bb)+\MET}\xspace}
\newcommand{\monoz}{\ensuremath{Z+\MET}\xspace}
\newcommand{\monozll}{\ensuremath{Z(\ell\ell)+\MET}\xspace}
\newcommand{\monozhad}{\ensuremath{Z(\textrm{had})+\MET}\xspace}

% Software/Program/Model names
\newcommand{\mg}{\textsc{MadGraph~5}\xspace}
\newcommand{\mgamcnlo}{MG5\_aMC@NLO\xspace}
\newcommand{\dmsimp}{\textsc{DMsimp}\xspace}
\newcommand{\maddm}{\textsc{MadDM}\xspace}
\newcommand{\hdma}{\ensuremath{\textrm{2HDM+a}}\xspace}

% Misc
\newcommand{\GamA}{\ensuremath{\Gamma_{A}}\xspace}
\newcommand{\Uli}{\color{red}}
\definecolor{cerulean}{RGB}{44,150,207}
\newcommand{\ATLASComments}{\color{cerulean}}
\newcommand{\bra}[1]{\langle #1|}
\newcommand{\ket}[1]{|#1\rangle}
\newcommand{\sens}{\mathcal{S}\xspace}
\newcommand{\senstot}{\mathcal{S}_\textrm{tot}\xspace}
\newcommand{\mathsc}[1]{\text{\textsc{#1}}}
\newcommand{\vev}[1]{\langle {#1} \rangle}

\DeclareMathOperator{\arccot}{arccot}

\def\be   {\begin{equation}}   \def\ee   {\end{equation}}
\def\ba   {\begin{array}}      \def\ea   {\end{array}}
\def\bea  {\begin{eqnarray}}   \def\eea  {\end{eqnarray}}
\def\bean {\begin{eqnarray*}}  \def\eean {\end{eqnarray*}}
\def\nn{\nonumber}


\allowdisplaybreaks

%DIF PREAMBLE EXTENSION ADDED BY LATEXDIFF
%DIF UNDERLINE PREAMBLE %DIF PREAMBLE
\RequirePackage[normalem]{ulem} %DIF PREAMBLE
\RequirePackage{color}\definecolor{RED}{rgb}{1,0,0}\definecolor{BLUE}{rgb}{0,0,1} %DIF PREAMBLE
\providecommand{\DIFadd}[1]{{\protect\color{blue}\uwave{#1}}} %DIF PREAMBLE
\providecommand{\DIFdel}[1]{{\protect\color{red}\sout{#1}}}                      %DIF PREAMBLE
%DIF SAFE PREAMBLE %DIF PREAMBLE
\providecommand{\DIFaddbegin}{} %DIF PREAMBLE
\providecommand{\DIFaddend}{} %DIF PREAMBLE
\providecommand{\DIFdelbegin}{} %DIF PREAMBLE
\providecommand{\DIFdelend}{} %DIF PREAMBLE
%DIF FLOATSAFE PREAMBLE %DIF PREAMBLE
\providecommand{\DIFaddFL}[1]{\DIFadd{#1}} %DIF PREAMBLE
\providecommand{\DIFdelFL}[1]{\DIFdel{#1}} %DIF PREAMBLE
\providecommand{\DIFaddbeginFL}{} %DIF PREAMBLE
\providecommand{\DIFaddendFL}{} %DIF PREAMBLE
\providecommand{\DIFdelbeginFL}{} %DIF PREAMBLE
\providecommand{\DIFdelendFL}{} %DIF PREAMBLE
%DIF END PREAMBLE EXTENSION ADDED BY LATEXDIFF

\def\bm#1{\mbox{\boldmath$#1$\unboldmath}} 

\begin{document}
\title{\begin{boldmath} \huge Summarizing experimental sensitivities of collider experiments to Dark Matter models and comparison to other experiments \vspace{7mm} \end{boldmath}}

%%%%%%

%Add your name here!

\author[1]{Your name here}
\affiliation[1]{Your institution here}


\abstract{
Plots summarizing the constraints on Dark Matter (DM) models can help visualize synergies between different searches for the same kind of experiment, as well as between different experiments. In this document, we summarise the potential reach of various future collider facilities in the context of simplified dark matter searches. This work lives within EF10 but draws upon contributed inputs from a wide range of Snowmass subgroups. We take as a starting point the plots currently made for LHC searches and recommended by the Dark Matter Working Group, also used for the BSM and Dark Matter chapters of the European Strategy Briefing Book. We are in discussion with the cosmic frontier regarding dark matter complementarity.
}  

\maketitle

%\newpage 


\vskip10pt


%%%%%%%%%%%%%%%%%%%%%%%%%%%%%%%%%%%%%%%%%%%%%%%%%%%%%%%%%%%%%%%%%%%%
%%%%%%%%%%%%%%%%%%%%%%%%%%%%%%%%%%%%%%%%%%%%%%%%%%%%%%%%%%%%%%%%%%%%
%%%%%%%%%%%%%%%%%%%%%%%%%%%%%%%%%%%%%%%%%%%%%%%%%%%%%%%%%%%%%%%%%%%%

\section{Introduction}
\label{sec:introduction}

Boyu, could you copy the LOI here? It would make a good introduction.

%%%%%%%%%%%%%%%%%%%%%%%%%%%%%%%%%%%%%%%%%%%%%%%%%%%%%%%%%%%%%%%%%%%%
\section{Theoretical framework}
\label{sec:models}

Should comment on and cite the DMWG paper so it is clear what we are talking about

Add subsection on the relationship to dark photons once that is understood

Comment on rescaling, if using

%%%%%%%%%%%%%%%%%%%%%%%%%%%%%%%%%%%%%%%%%%%%%%%%%%%%%%%%%%%%%%%%%%%%
\section{Future collider facilities}
\label{sec:colliders}

Summarise the inputs we received and what their parameters are


%%%%%%%%%%%%%%%%%%%%%%%%%%%%%%%%%%%%%%%%%%%%%%%%%%%%%%%%%%%%%%%%%%%%
\section{Summary plots}
\label{sec:plots}

\subsection{Vector and axial-vector collider limits}

Plots 1 and 2 with all variations

\subsection{A/A-V comparisons to direct and indirect detection limits}

Plots 3 \& 4

\subsection{Limits in dark photon model}

Plot 5

\subsection{[Unlikely to include] Higgs-like and Higgs portal models}


%%%%%%%%%%%%%%%%%%%%%%%%%%%%%%%%%%%%%%%%%%%%%%%%%%%%%%%%%%%%%%%%%%%%
\section{Conclusion}


%%%%%%%%%%%%%%%%%%%%%%%%%%%%%%%%%%%%%%%%%%%%%%%%%%%%%%%%%%%%%%%%%%%%
%%%%%%%%%%%%%%%%%%%%%%%%%%%%%%%%%%%%%%%%%%%%%%%%%%%%%%%%%%%%%%%%%%%%
%%%%%%%%%%%%%%%%%%%%%%%%%%%%%%%%%%%%%%%%%%%%%%%%%%%%%%%%%%%%%%%%%%%%

\acknowledgments 

[To be updated]

%%%%%%%%%%%%%%%%%%%%%%%%%%%%%%%%%%%%%%%%%%%%%%%%%%%%%%%%%%%%%%%%%%%
%%%%%%%%%%%%%%%%%%%%%%%%%%%%%%%%%%%%%%%%%%%%%%%%%%%%%%%%%%%%%%%%%%%%
%%%%%%%%%%%%%%%%%%%%%%%%%%%%%%%%%%%%%%%%%%%%%%%%%%%%%%%%%%%%%%%%%%%%

%\appendix


\newpage 

\bibliography{DMProjections-draft}
\bibliographystyle{JHEP}



\end{document}
